\documentclass[margin,line]{res}

\usepackage{url}
\usepackage{hyperref}


%\usepackage{marvosym}
%\usepackage{url,parskip}            %formatting
%
%%Graphics - Colors
%\RequirePackage{color,graphicx}
%\usepackage[usenames,dvipsnames]{xcolor}
%%better formatting of the A4 page
%\usepackage[big]{layaureo}
%%An alternative to Layaureo can be usepackage{fullpage}
%
%\usepackage{supertabular}       %for Grades
%\usepackage{titlesec}           %custom section
%
%%Setup hyperref package, and colours for links
%\usepackage{hyperref}
%\definecolor{linkcolour}{rgb}{0,0.2,0.6}
%\hypersetup{colorlinks,breaklinks, urlcolor=linkcolour,
%linkcolor=linkcolour}

%\usepackage{hyperref}

%\hypersetup{
    %bookmarks=true,         % show bookmarks bar?
    %unicode=false,          % non-Latin characters in Acrobat�s bookmarks
    %pdftoolbar=true,        % show Acrobat�s toolbar?
    %pdfmenubar=true,        % show Acrobat�s menu?
    %pdffitwindow=false,     % window fit to page when opened
    %pdfstartview={FitH},    % fits the width of the page to the window
    %pdftitle={My title},    % title
    %pdfauthor={Author},     % author
    %pdfsubject={Subject},   % subject of the document
    %pdfcreator={Creator},   % creator of the document
    %pdfproducer={Producer}, % producer of the document
    %pdfkeywords={keyword1} {key2} {key3}, % list of keywords
    %pdfnewwindow=true,      % links in new window
    %colorlinks=false,       % false: boxed links; true: colored links
    %linkcolor=red,          % color of internal links
    %citecolor=green,        % color of links to bibliography
    %filecolor=magenta,      % color of file links
    %urlcolor=cyan           % color of external links
%}

\oddsidemargin -.5in \evensidemargin -.5in \textwidth=6.0in
\itemsep=0in
\parsep=0in

\newenvironment{list1}{
  \begin{list}{\ding{113}}{%
      \setlength{\itemsep}{0in}
      \setlength{\parsep}{0in} \setlength{\parskip}{0in}
      \setlength{\topsep}{0in} \setlength{\partopsep}{0in}
      \setlength{\leftmargin}{0.17in}}}{\end{list}}
\newenvironment{list2}{
  \begin{list}{$\bullet$}{%
      \setlength{\itemsep}{0in}
      \setlength{\parsep}{0in} \setlength{\parskip}{0in}
      \setlength{\topsep}{0in} \setlength{\partopsep}{0in}
      \setlength{\leftmargin}{0.2in}}}{\end{list}}


\begin{document}

\name{Jun Liu \vspace*{.1in}}

\begin{resume}
\section{\sc Contact Information}
\vspace{.05in}
\begin{tabular}{@{}p{3.7in}p{3in}}
{Email:}  JunLiu.NT@Gmail.com  & 506 Gallberry Dr, Cary, NC 27519 \\
{Homepage:} {\small \url{https://sites.google.com/site/junliupage/}}    &  {Phone:}  1-609-216-0395 (cell) \\
\end{tabular}
%\begin{tabular}{@{}p{5in}}
%{Email:}  JunLiu.NT@Gmail.com \\
%{Personal page:}  \url{http://parnec.nuaa.edu.cn/jliu/} \\
%\end{tabular}



%\vspace{0.05in}

\section{\sc Research Interests}

$\bullet$ Sparse Learning

\vspace{-0.05in}

I am the principal developer for the SLEP (Sparse Learning with
Efficient Projections) package, which provides efficient solvers for
quite a few sparse learning models including Lasso, group Lasso,
fused Lasso, tree structured group Lasso, overlapping group Lasso,
sparse inverse covariance estimation, nuclear norm regularization,
and so on. Since its release at the end of 2009, the SLEP package has
been cited over 500 times according to Google scholar. I have published
over 50 papers that have received over 4,000 citations according to Google scholar. 



$\bullet$ Parallel MR Imaging


\vspace{-0.05in}


At Siemens Corporate Research, I focused on Parallel MR (Magnetic Resonance) reconstruction
using the compressed sensing technique. My major contributions included:
\begin{itemize}
	\item 1. the development of a novel and robust coil profile estimation approach which plays a key role
in the SENSE-type MR reconstruction,

	\item 2. the design of an effective and powerful approach for reconstruction of dynamic MR images
by sampling less than $\frac{1}{6}$ of k-space data, enabling the acquisition of high spatial/temporal resolution real time MR images,

	\item 3. building a portfolio of algorithms for parallel MR reconstruction with non-smooth regulariza-
tion; and

	\item 4. inventing a novel MR reconstruction approach for the restricted Field-Of-View scenario.

\end{itemize}

My work has been implemented into the Siemens Image Calculation Environment (ICE) environment
and demonstrated at the Siemens MR scanner, and was demonstrated at the Siemens Healthcare
IDEA Meeting at the ISMRM 2012, Melbourne, Victoria, Australia.
Some highlights of this work:
\begin{itemize}
	\item 1. A Siemens CT RTC ICV Award (In recognition of excellence in the category: Outstanding
innovation jointly achieved with a lead customer) was given to this work.
	\item 2. We won the second place for the Recon Challenge on 2D free-breathing Cardiac Cine at Recon
challenge 2013 \url{http://www.ismrm.org/workshops/Data13/challenge.htm}.
	\item 3. The MR reconstruction approach used in the news \url{http://www.brisbanetimes.com.au/
queensland/cadel-pedals-for-heart-disease-20131111-2xblx.html} was my work.
	\item 4. Siemens is the first company to release the compressed sensing MRI acceleration, and FDA
has cleared this revolutionary Compressed Sensing technology, see \url{https://usa.healthcare.
siemens.com/press/pressreleases/healthcare-news-2017-02-21-1}.
	\item 5. I have been nine granted patents on this work as of 2018.
\end{itemize}


$\bullet$ Feature Extraction

\vspace{-0.05in}

I have conducted quite a few research on the feature extraction
approaches such as Principal Component Analysis (PCA) and Linear
Discriminant Analysis (LDA). By studying and analyzing quite a few
variants for PCA and LDA, I have revealed the relationships among
these approaches and developed efficient alternatives. In addition,
I have studied the two-dimensional PCA and come up with new
approaches for feature extraction.

\newpage

$\bullet$ Face Recognition


\vspace{-0.05in}

I have developed approaches for face recognition under different
facial variation such as expression, illumination and
occlusion. In particular, I have designed several algorithms
specialized for recognition from single image per person. I have
also developed an algorithm for face liveness detection, trying to
avoid spoofing the face recognition system with photograph or video.


%$\bullet$ Multi-task Learning
%
%
%\vspace{-0.05in}
%
%I have developed multi-task learning algorithms for performing joint
%feature selection across a group of related tasks. The aim is to
%make use of the shared information among related tasks for improved
%performance.


$\bullet$ Model Selection, Big Data, and Streaming Analytics


\vspace{-0.05in}

At SAS Institute Inc., I focused on developing efficient and easy-to-use
model selection approaches such as forward selection, backward selection,
Lasso selection, elastic net and group Lasso in PROC GLMSELECT.
In terms of efficiency, I have developed the Sasvi technique which
performs safe screening with variational inequalities. I have also developed
efficient algorithms for Principal Component Analysis (PCA) in the distributed setting for big data.

%$\bullet$ Alzheimer's Disease
%
%
%\vspace{-0.05in}
%
%Alzheimer��s disease (AD) is a fatal, neurodegenerative disorder
%characterized by progressive impairment of memory and other
%cognitive functions. I have studied the estimation of brain
%connectivity using sparse inverse covariance estimation. I have also
%studied modeling/predicting disease progression.

%\newpage

\section{\sc Education}
{Nanjing University of Aeronautics and Astronautics} \hfill Nanjing, P. R. China\\
\vspace*{-.1in}
\begin{list1}
\item[] Ph.D., Computer Science \hfill 2002.9-2007.11

\begin{list2}
\vspace*{.05in}
\item Dissertation Topic:  ``Research on Subspace Representation of Face Images''
\item Advisor:  Prof. Songcan Chen
\end{list2}
\vspace*{.05in}
%\item[] Note: direct pursuit for Ph.D. nominated from the M.S. candidates,
%September,
%2002
\end{list1}

{Nantong University} \hfill Nantong, P. R. China\\
%{\em Department of Mathematics and Statistics}
\vspace*{-.1in}
\begin{list1}
\item[] Bachelor, Computer Science \hfill 1998.9-2002.7
\end{list1}


\vspace{0.05in}

\section{\sc Professional Appointments}


%\vspace{-.1cm}
%{\em Teaching Assistant} \hfill {\bf April, 2006  - November, 2007}\\
%Duties at various times have included instructing undergraduates
%composing thesis, and leading weekly seminar discussions.


\vspace{-.1cm}
{\em Lecture} \hfill 2007.11-2009.06\\
The Computer Science Department, \\ Nanjing University of Aeronautics
and Astronautics

\vspace{-.1cm}
{\em Associate Professor} \hfill 2009.06-2010.05\\
The Computer Science Department,\\ Nanjing University of Aeronautics
and Astronautics

\vspace{-.1cm}
{\em Postodc} \hfill 2008.02-2010.11\\
The Biodesign Institute and the Computer Science and Engineering Department,\\
Arizona State University


\vspace{-.1cm}
{\em Research Scientist} \hfill 2010.12-2011.02\\
The Biodesign Institute and the Computer Science and Engineering Department,\\
Arizona State University

\vspace{-.1cm}
{\em Research Scientist} \hfill 2011.02-2012.12\\
Siemens Corporate Research

\vspace{-.1cm}
{\em Research Statistician Developer} \hfill 2013.01-2014.03\\
SAS Institute Inc.

\newpage

\vspace{-.1cm}
{\em Senior Research Statistician Developer} \hfill 2014.04-2017.03\\
SAS Institute Inc.

\vspace{-.1cm}
{\em Principal Research Statistician Developer} \hfill 2017.03-2017.07\\
SAS Institute Inc.


\vspace{-.1cm}
{\em Principal Data Scientist} \hfill 2017.09-present\\
Infinia ML Inc.

%\vspace{0.05in}

%\section{\sc NSFC Award}
%
%Large-scale sparse learning and its applications \hfill 2009\\
%Principal Investigator\\ National Science Foundation of China

\section{\sc Honors and Awards}

%National College English Contest (First Price) \hfill 2000

\vspace*{-.1cm} China Aviation Supplies Import \& Export Group
Corporation (CASC) Prize \hfill 2006

\vspace*{-.1cm} The 13th International Conference on Neural
Information Processing (ICONIP) Travel Award \hfill 2006

\vspace*{-.1cm} The 21st International Joint Conference on
Artificial Intelligence (IJCAI) Travel Award \hfill 2009

\vspace*{-.1cm} The 23rd Annual Conference on Neural Information
Processing Systems (NIPS) Travel Award \hfill 2009

\vspace*{-.1cm} The 25th Conference on Uncertainty in Artificial
Intelligence (UAI) Travel Award \hfill 2009

%\vspace*{-.1cm} The 26th International Conference on Machine
%Learning (ICML) Travel Award, 2009

\vspace*{-.1cm} Excellent Ph.D. Thesis of Jiansu Province, P. R.
China \hfill 2009


\vspace*{-.1cm} The 24th Annual Conference on Neural Information
Processing Systems (NIPS) Travel Award \hfill 2010


\vspace*{-.1cm}  Siemens CT RTC ICV Award (In recognition of excellence in the category: \\ Outstanding innovation jointly achieved with a lead customer) \\ for MR Sparse Reconstruction
\hfill  2012

\vspace*{-.1cm}  2nd place for the Recon Challenge on 2D free-breathing Cardiac Cine \\
(Recon challenge page: \url{http://www.ismrm.org/workshops/Data13/challenge.htm}) \hfill 2013

%\section{\sc Research { } Project}
%
%%60905035:
%
%Large-scale sparse learning and its applications \hfill 2010-2012\\
%Principal Investigator\\ National Science Foundation of China
%
%%61035003:
%
%Cloud computing-based massive data mining \hfill 2011-2014\\
%Co-Principal Investigator\\ National Science Foundation of China
%
%
%%60773061:
%
%Classifier localized regular Design Technology \hfill 2008-2010\\
%Co-Principal Investigator\\ National Science Foundation of China
%
%%60773060:
%
%Machine learning based single sample face
%recognition in real environment  \hfill 2008-2010\\
%Co-Principal Investigator\\ National Science Foundation of China
%
%
%%6050500:
%
%Semi-supervised learning and its applications \hfill 2006-2008\\
%Co-Principal Investigator\\ National Science Foundation of China
%
%
%%60473035:
%
%Enhanced classification and linear discriminant analysis to promote research \hfill 2005-2007\\
%Co-Principal Investigator\\ National Science Foundation of China



%\vspace{0.1in}


\section{\sc Granted Patents}

[11] \underline{J. Liu}, R. Zhang, Y. Xu, J. Griffin. ``Enhanced power method on an electronic device," US Patent 9,684,538.

[10] \underline{J. Liu}, J. Rapin, A. Lefebvre, M. Nadar, T. Chang.
``Efficient redundant haar minimization for parallel {MRI} reconstruction," US Patent 9,632,156.

[9] \underline{J. Liu}, Q. Wang, M. Nadar, M. Zenge, and E. Mueller ``Dynamic Image Reconstruction with 
Tight Frame Learning," US Patent 9,453,895.

[8] A. Lefebvre, \underline{J. Liu}, E. Mueller, M. S Nadar, M. Schmidt, M. Zenge, and Q. Wang ``MRI recon-
struction with incoherent sampling and redundant haar wavelets," US Patent 9,396,562.

[7] M. S Nadar, S. Martin, A. Lefebvre, and \underline{J. Liu}. ``Multi-GPU FISTA Implementation for MR
Reconstruction with Non-Uniform K-Space Sampling," US Patent 9,466,102.

[6] N. Chesneau, N. Janardhanan, \underline{J. Liu}, M. S Nadar, Q. Wang, and Z. Yang. ``MRI reconstruction
with motion-dependent regularization," US Patent 9,482,732.

[5] Z. Zhao, \underline{J. Liu}, and J. Cox. ``Acceleration Of Sparse Support Vector Machine Training Through
Safe Feature Screening," US Patent 9,495,647.

[4] A. Lefebvre, A. Loewe, M. S Nadar, and \underline{J. Liu}. ``Zero communication block partitioning," US
Patent 9,286,648.

[3] \underline{J. Liu}, Z. Yang, M. S Nadar, N. Janardhanan, M. Zenge, E. Mueller, Q. Wang, and A. Loewe
``Multi-stage magnetic resonance reconstruction for parallel imaging applications,"? US Patent 9,097,780.

[2] \underline{J. Liu}, A. Lefebvre, and M. Nadar. ``Alternating direction of multipliers method for parallel MRI
reconstruction," US Patent 8,879,811.

[1] \underline{J. Liu}, J. Rapin, A. Lefebvre, M. Nadar, T. Chang, M. Zenge, E. Muller. ``Image reconstruction
using redundant haar wavelets," US Patent 8,948,480.



\section{\sc Nonprovisional Patent Applications}

[2] \underline{J. Liu}, and Z. Zhao.
``Linear Regression Using Safe Screening Techniques," US Patent App. 14/571,224.

[1] \underline{J. Liu}, H. Xue, M. D. Nickel, T. Chang, M. Nadar, A. Lefebvre, E. Mueller, Q. Wang, Z. Yang, N. Janardhanan, and M. Zenge.
``Eigen-vector approach for coil sensitivity maps estimation," US Patent App. 13/779,999.



\vspace{0.3in}

\section{\sc Publications}

{\em Peer-Reviewed Conference Papers (reverse chronological order)}

[36]  \underline{J. Liu}, Z. Zhao, R. Zhang, and Y. Xu. ``Successive Ray Refinement and Its Application to
Coordinate Descent for Lasso," Intelligent Data Engineering and Automated Learning (IDEAL)
2016.

[35] T. Yang,  \underline{J. Liu}, P. Gong, R. Zhang, X. Shen, and J. Ye. ``Absolute Fused Lasso and Its Appli-
cation to Genome-Wide Association Studies," ACM SIGKDD Conference on Knowledge Discovery and Data Mining (SIGKDD) 2016

[34] J. Wang, J. Zhou, \underline{J. Liu}, P. Wonka, and J. Ye. 
``A Safe Screening Rule for Sparse Logistic Regression," 
\emph{Advances in Neural Information Processing Systems} (NIPS) 2014.

[33] \underline{J. Liu}, Z. Zhao, J. Wang, and J. Ye. 
``Safe Screening With Variational Inequalities and Its Applicaiton to LASSO," 
\emph{International Conference on Machine Learning} (ICML) 2014.

[32] Z. Zhao, \underline{J. Liu}, and J. Cox. 
``Safe and Efficient Screening For Sparse Support Vector Machine,"
\emph{ACM SIGKDD Conference on Knowledge Discovery and Data Mining} (SIGKDD) 2014.

[31] Z. Zhao, \underline{J. Liu}, and J. Cox. 
``Accelerating Model Selection with Safe Screening for L1-Regularized L2-SVM,"
\emph{European Conference on Machine Learning and Principles and Practice of Knowledge Discovery in Databases} (ECML/PKDD) 2014.

[30] Q. Wang, \underline{J. Liu}, N. Janardhanan, M. Zenge, E. Mueller, and M. Nadar.
``Tight Frame Learning For Cardiovascular MRI," 
\emph{International Symposium on Biomedical Imaging} (ISBI) 2013.


[29] D. Zhang, \underline{J. Liu}, and D. Shen.
``Temporally-Constrained Group Sparse Learning for Longitudinal Data
Analysis," \emph{International Conference on Medical Image Computing
and Computer Assisted Intervention} (MICCAI) 2012.

[28] Y. Hu, D. Zhang, \underline{J. Liu}, J. Ye, and X. He.
``Accelerated Singular Value Thresholding for Matrix Completion,"
\emph{ACM SIGKDD Conference on Knowledge Discovery and Data Mining}
(SIGKDD) 2012.

[27] S. Ji, W. Zhang, and \underline{J. Liu}. ``A Sparsity-Inducing
Formulation for Evolutionary Co-Clustering," \emph{ACM SIGKDD
Conference on Knowledge Discovery and Data Mining} (SIGKDD) 2012.


[26] J. Zhou, \underline{J. Liu}, V. Narayan, and J. Ye. ``Modeling
Disease Progression via Fused Sparse Group Lasso," \emph{ACM SIGKDD
Conference on Knowledge Discovery and Data Mining} (SIGKDD) 2012.

[25] K. Zhang, L. Lan, \underline{J. Liu}, A. Rauber, and F.
Moerchen. ``Inductive Kernel Low-rank Decomposition with Priors,"
\emph{ International Conference on Machine Learning} (ICML) 2012.


[24] J. Gao, M. Paul, and \underline{J. Liu}. ``The Image Matting
Method with Regularized Matte," \emph{IEEE International Conference
on Multimedia and Expo} (ICME) 2012.


[23] \underline{J. Liu}, L. Sun and J. Ye. ``Projection onto A
Nonnegative Max-Heap," \emph{Advances in Neural Information
Processing Systems} (NIPS) 2011.

[22] L. Yuan, \underline{J. Liu} and J. Ye. Efficient Methods for
Overlapping Group Lasso, \emph{Advances in Neural Information
Processing Systems} (NIPS) 2011.

[21] J. Zhou, L. Yuan, \underline{J. Liu} and J. Ye. ``A Multi-Task
Learning Formulation for Predicting Disease Progression," \emph{ACM
SIGKDD Conference on Knowledge Discovery and Data Mining} (KDD)
2011.

[20] \underline{J. Liu} and J. Ye. ``Moreau-Yosida Regularization
for Grouped Tree Structure Learning," \emph{Advances in Neural
Information Processing Systems} (NIPS) 2010.

[19] \underline{J. Liu}, L. Yuan, and J. Ye. ``An Efficient
Algorithm for a Class of Fused Lasso Problems," \emph{ACM SIGKDD
Conference on Knowledge Discovery and Data Mining} (KDD) 2010.

[18] H. Liu, J. Zhang, X. Jiang, and \underline{J. Liu}. ``The Group
Dantzig Selector," \emph{International Conference on Artificial
Intelligence and Statistics} (AI \& Statistics) 2010.

[17] X. Tan, Y. Li, \underline{J. Liu}, and L. Jiang. ``Face
Liveness Detection from A Single Image with Sparse Low Rank Bilinear
Discriminative Model," \emph{European Conference on Computer Vision}
(ECCV) 2010.

[16] \underline{J. Liu}, S. Ji, and J. Ye. ``Multi-Task Feature
Learning Via Efficient $\ell_{2,1}$-Norm Minimization,"
\emph{Uncertainty in Artificial Intelligence} (UAI) 2009.

[15] \underline{J. Liu}, J. Chen, and J. Ye. ``Large-Scale Sparse
Logistic Regression," \emph{ACM SIGKDD Conference on Knowledge
Discovery and Data Mining} (KDD) 2009.

[14] \underline{J. Liu} and J. Ye. ``Efficient Euclidean Projections
in Linear Time," \emph{International Conference on Machine Learning}
(ICML) 2009.

[13] \underline{J. Liu}, J. Chen, S. C. Chen, and J. Ye. ``Learning
the Optimal Neighborhood Kernel for Classification,"
\emph{International Joint Conference on Artificial Intelligence}
(IJCAI) 2009.

[12] J. Chen, L. Tan, \underline{J. Liu}, and J. Ye. ``A Convex
Formulation for Learning Shared Structures from Multiple Tasks,"
\emph{International Conference on Machine Learning} (ICML) 2009.

[11] L. Sun, \underline{J. Liu}, J. Chen, and J. Ye. ``Efficient
Recovery of Jointly Sparse Vectors," \emph{ Annual Conference on
Neural Information Processing Systems} (NIPS) 2009.

[10] S. Huang, J. Li, L. Sun, \underline{J. Liu}, T. Wu, K. Chen, A.
Fleisher, E. Reiman, and J. Ye. ``Learning Brain Connectivity of
Alzheimer's Disease from Neuroimaging Data," \emph{Conference on
Neural Information Processing Systems} (NIPS) 2009.

[9] L. Sun, R. Patel, \underline{J. Liu}, K. Chen, T. Wu, J. Li, E.
Reiman, and J. Ye. ``Mining Brain Region Connectivity for
Alzheimer's Disease Study via Sparse Inverse Covariance Estimation,"
\emph{ACM SIGKDD Conference on Knowledge Discovery and Data Mining}
(KDD) 2009.

[8] X. Tan, L. Qiao,  W. Gao, and \underline{J. Liu}. ``Robust faces
manifold modeling: Most expressive Vs. most Sparse criterion,"
\emph{Subspace Workshop (in conjunction with ICCV)} 2009.

[7] \underline{J. Liu}, S. C. Chen, Z. H. Zhou, and X. Tan. ``Single
Image Subspace for Face Recognition," {\em IEEE International
Workshop on Analysis and Modeling of Faces and Gestures} (AMFG, In
conjunction with ICCV) 2007.

[6] X. Tan, S. C. Chen, Z. H. Zhou and \underline{J. Liu}.
``Learning Non-Metric Partial Similarity Based on Maximal Margin
Criterion," {\em IEEE Computer Society Conference on Computer Vision
and Pattern Recognition} (CVPR) 2006.

[5] X. Tan, \underline{J. Liu} and S. C. Chen. ``Recognition From a
Single Sample Per Person With Multiple SOM Fusion," {\em
International Symposium on Neural Networks} (ISNN) 2006.

[4] \underline{J. Liu} and S. C. Chen. ``Resampling LDA/QR and
PCA+LDA for Face Recognition," {\em Australian Joint Conference on
Artificial Intelligence} (AJCAI) 2005.

[3] X. Tan, \underline{J. Liu} and S. C. Chen. ``Weighted SOM-Face:
Selecting Local Features for Recognition From Individual Face
Image," {\em Intelligent Data Engineering and Automated Learning}
(IDEAL) 2005.

[2] D. Zhang, S. C. Chen and \underline{J. Liu}. ``Representing
Image Matrices: Eigenimages Versus Eigenvectors," {\em International
Symposium on Neural Networks} (ISNN) 2005.

[1] \underline{J. Liu}, S. C. Chen and Z. H. Zhou. ``Progressive
Principal Component Analysis," {\em International Symposium on
Neural Networks} (ISNN) 2004.




\vspace{0.3in}


{\em Peer-Reviewed  Journal Papers (reverse chronological order)}

[20] L. Lan, K. Zhang, H. Ge, W. Cheng, \underline{J. Liu}, A. Rauber, X.L. Li, J. Wang, H. Zha. ``Low-rank decomposition meets kernel learning: A generalized Nystr�m method," {\em Artificial Intelligence}, 250 (2017): 1-15.

[19] B. Jie, M. Liu, \underline{J. Liu}, D. Zhang, and D. Shen. ``Temporally-Constrained Group Sparse Learning
for Longitudinal Data Analysis in Alzheimers Disease,? {\em IEEE Transactions on Biomedical Engineering}, 2016.

[18] J. Zhou, \underline{J. Liu}, V. Narayan, and J. Ye. ``Modeling disease progression via multi-task learning," 
{\em Neuroimage}, 78 (2013): 233-248.

[17] X. Liu, J. Yin, Lei Wang, L. Liu, \underline{J. Liu}, C. Hou
and J. Zhang. ``An Adaptive Approach to Learning Optimal
Neighborhood Kernels," {\em IEEE Transactions on Systems, Man, and
Cybernetics, Part B}, 43, no. 1 (2013): 371-384.

[16] L. Yuan, \underline{J. Liu}, and J. Ye. ``Efficient Methods for Overlapping Group Lasso," 
{\em IEEE Transactions on Pattern Analysis and Machine Intelligence}, 35, no. 9 (2013): 2104-2116.

[15] J. Chen, L. Tan, \underline{J. Liu}, and J. Ye. ``A Convex Formulation for
Learning Shared Structures from Multiple Tasks," {\em IEEE
Transactions on Pattern Analysis and Machine Intelligence}, 35, no. 5 (2013): 1025-1038.

[14] J. Ye and \underline{J. Liu}. ``Sparse Methods for Biomedical
Data," {\em ACM SIGKDD Explorations}, 14, no. 1 (2012): 4-15.

[13] J. Liu, \underline{J. Liu}, P. Wonka, and J. Ye. ``Sparse
Non-negative Tensor Factorization Using Columnwise Coordinate
Decent," {\em Pattern Recognition}, 45, no. 1 (2012): 649-656.

[12] \underline{J. Liu}, S. C. Chen, Z. H. Zhou and X. Tan.
``Generalized Low Rank Approximations of Matrices Revisited," {\em
IEEE Transactions on Neural Networks}, 21, no. 4 (2010): 621-632.


[11] Y. L. Zhu, \underline{J. Liu}, S. C. Chen. ``Semi-Random
Subspace Method for Face Recognition," {\em Image \& Vision
Computing}, 27, no. 9 (2009): 1358-1370.

[10] X. Tan, S. C. Chen, Z. H. Zhou, and \underline{J. Liu}. ``Face
Recognition under Occlusions and Variant Expressions with Partial
Similarity," {\em IEEE Transactions on Information Forensics \&
Security}, 4, no.~2 (2009): 217-230.


[9] \underline{J. Liu}, S. C. Chen and X. Tan. ``Fractional order
Singular Value Decomposition Representation for Face Recognition,"
{\em Pattern Recognition} 41, no. 1 (2008): 378-395.

[8] \underline{J. Liu}, S. C. Chen and X. Tan. ``A Study on Three
Linear Discriminant Analysis Based Methods in Small Sample Size
Problem," {\em Pattern Recognition}, 41, no. 1 (2008): 102-116.


[7] Z. Wang, S. C. Chen, \underline{J. Liu}, and D. Zhang. ``Pattern
Representation in Feature Extraction and Classification- Matrix
Versus Vector," {\em IEEE Transactions on Neural Networks}, 19, no.
5, (2008): 758-769.


[6] \underline{J. Liu}, S. C. Chen, X. Tan and D. Zhang. ``Comments
on ``Efficient and Robust Feature Extraction by Maximum Margin
Criterion''," {\em IEEE Transactions on Neural Networks}, 18, no. 6
(2007): 1862-1864.

[5] \underline{J. Liu}, S. C. Chen, X. Tan and D. Zhang. ``Efficient
Pseudo-Inverse Linear Discriminant Analysis and Its Nonlinear Form
for Face Recognition." {\em International Journal of Pattern
Recognition and Artificial Intelligence}, 21, no. 8 (2007):
1265-1278.


[4] \underline{J. Liu} and S. C. Chen.  ``Discriminant Common
Vectors versus Neighbourhood Components Analysis and Laplacianfaces:
A Comparative Study in Small Sample Size Problem," {\em Image and
Vision Computing} 24, no. 3 (2006): 249-262.

[3] \underline{J. Liu} and S. C. Chen. ``Non-Iterative Generalized
Low Rank Approximation of Matrices," {\em Pattern Recognition
Letters} 27, no. 9 (2006): 1002-1008.


[2] X. Tan, \underline{J. Liu} and S. C. Chen. ``Sub-Intrapersonal
Space Analysis for Face Recognition," {\em Neurocomputing} 69, no.
13-15 (2006): 1796-801.


[1] S. C. Chen, \underline{J. Liu} and Z. H. Zhou. ``Making FLDA
Applicable to Face Recognition With One Sample Per Person," {\em
Pattern Recognition} 37, no. 7 (2004): 1553-1555.



\vspace{0.3in}


{\em Conference Abstracts (reverse chronological order)}



[9] M. Zenge, A. Lefebvre, C. Forman, R. Grimm, J. Hutter, \underline{J. Liu}, N. Janardhanan, and M. Nadar.
``IRecon – Introducing a Standardized Interface into the Siemens Image Reconstruction Environment,"
\emph{ISMRM Data Sampling \& Image Reconstruction Workshop in Sedona}, 2013.
 

[8] \underline{J. Liu}, A. Lefebvre, M. S. Nadar, M. Zenge, M. Schmidt, and E. Mueller.
``2D bSSFP Real-time Cardiac CINE-MRI: Compressed Sensing Featuring Weighted RedundantHaar Wavelet Regularization in Space and Time,"
\emph{SCMR 2013 SCIENTIFIC SESSIONS}. 2013.

[7] M. Schmidt, E. Mueller, M. Zenge, \underline{J. Liu}, A. Lefebvre, M. S. Nadar, and O. Ekinci.
``Novel Highly Accelerated Real-Time CINE-MRI featuring Compressed Sensing with k-t Regularization in Comparison to TSENSE Segmented and Real-Time Cine Imaging,"
\emph{SCMR 2013 SCIENTIFIC SESSIONS}. 2013. 

[6] \underline{J. Liu}, A. Loewe, M. Zenge, A. Lefebvre, E. Mueller, and M. S. Nadar.
``Elora: Enforcing Low Rank for Parallel MR Reconstruction,"
\emph{ISMRM} 2013. 

[5] Q. Wang, \underline{J. Liu}, M. Zenge, N. Janardhanan, E. Mueller, and M. S. Nadar.
``An Eigen-Vector Approach for Coil Sensitivity Estimation in the 3D Scenario,"
\emph{ISMRM} 2013. 

[4] Q. Wang, \underline{J. Liu}, N. Janardhanan,  and M. S. Nadar.
``Cardiovascular MRI Reconstruction with Data-Driven Sparsifying Transform," 
\emph{ISMRM} 2013.

[3] Q. Wang, \underline{J. Liu}, Z. Yang, N. Chesneau, M. Zenge, M. Schmidt, N. Janardhanan, E. Mueller, and M. S. Nadar.
``Motion-Dependent L1 Minimization for Dynamic Cardiac MRI Reconstruction," \emph{ISMRM} 2013.

[2] \underline{J. Liu}, J. Rapin, T. Chang,  A. Lefebvre, M. Zenge,
E. Mueller, and M. S. Nadar. Dynamic cardiac MRI reconstruction with
weighted redundant Haar wavelets. \emph{ISMRM} 2012.

[1] \underline{J. Liu}, J. Rapin, T. Chang, P. Schmitt, X. Bi, A.
Lefebvre, M. Zenge, E. Mueller, and M. S. Nadar. ``Regularized
reconstruction using redundant Haar wavelets: A means to achieve
high under-sampling factors in non-contrast-enhanced 4D MRA," \emph{ISMRM} 2012.



%\newpage

%\vspace{+.2cm}
%{\em Preprint}
%
%[1] \underline{J. Liu} and J. Ye. ``Efficient $\ell_1$/$\ell_q$-norm
%Regularization," \emph{arXiv:1009.4766v1}, 2010.
%
%[2] \underline{J. Liu} and J. Ye. ``Fast Overlapping Group Lasso,"
%\emph{arXiv:1009.0306v1}, 2010.




\section{\sc Professional Activities}


{\em Tutorial}

\begin{list2}
\item J. Liu, S. Ji, and J. Ye. ``Mining Sparse Representations: Formulations, Algorithms, and Applications." SIAM Conference on Data Mining (SDM), Columbus, Ohi, USA,
2010.
\end{list2}


\vspace{-.15cm}

{\em Journal Reviewers}

\begin{list2}

\vspace{+.1cm}
\item
Computational Statistics and Data Analysis

\vspace{+.1cm}
\item
Frontiers of Computer Science in China

\vspace{+.1cm}
\item
IEEE Transactions on Image Processing

\vspace{+.1cm}
\item
IEEE Transactions on Circuits and Systems for Video Technology

\vspace{+.1cm}
\item
IEEE Transactions on Knowledge and Data Engineering

\vspace{+.1cm}
\item
IEEE Transactions on Neural Networks

\vspace{+.1cm}
\item
IEEE Transactions on Pattern Analysis and Machine Intelligence

\vspace{+.1cm}
\item
IEEE Transactions on Systems, Man, and Cybernetics, Part B

\vspace{+.1cm}
\item
Image and Vision Computing

\vspace{+.1cm}
\item
Information Science

\vspace{+.1cm}
\item
International Journal of Machine Learning Research

\vspace{+.1cm}
\item
International Journal of Pattern Recognition and Artificial
Intelligence

\vspace{+.1cm}
\item
International Journal of Software and Informatics

\vspace{+.1cm}
\item
Journal of Software and Informatics

\vspace{+.1cm}
\item
Journal of Computer Science and Technology

\vspace{+.1cm}
\item
Neurocomputing

\vspace{+.1cm}
\item
Pattern Analysis \& Applications

\vspace{+.1cm}
\item
Pattern Recognition Letters

\vspace{+.1cm}
\item
SCIENCE CHINA Information Sciences


\vspace{+.1cm}
\item
Signal Processing


\end{list2}


\vspace{+.2cm}

{\em Conference Reviewers or Program Committee Members}

\begin{list2}


\vspace{+.1cm}
\item
Advances in Neural Information Processing Systems


\vspace{+.1cm}
\item
ACM SIGKDD Conference on Knowledge Discovery and Data Mining



\vspace{+.1cm}
\item
International Conference on Machine Learning


\vspace{+.1cm}
\item
SIAM International Conference on Data Mining

\vspace{+.1cm}
\item
International Conference on Data Mining





%\vspace{+.1cm}
%\item
%European Conference on Machine Learning and Principles and Practice
%of Knowledge Discovery in Databases (ECML PKDD) 2010




\vspace{+.1cm}
\item
International Conference on Machine Learning and Applications

%\vspace{+.1cm}
%\item
%International Symposium on Neural Networks 2007

%\vspace{+.1cm}
%\item ACM SIGKDD Conference on Knowledge Discovery and Data Mining 2010
%(External Reviewer)
%
%\vspace{+.1cm}
%\item
%European Conference on Computer Vision 2010 (External Reviewer)


\end{list2}

 

\section{\sc Main Open-Source Softwares Developed}

I have developed the SLEP package, which provides functions for
quite a few sparse learning models:

\begin{list2}
\vspace{+.1cm}
\item $\ell_1$-norm Regularized / Constrained Optimization

\vspace{+.1cm}
\item $\ell_1/\ell_q$-norm Regularized Optimization

\vspace{+.1cm}
\item Fused Lasso

\vspace{+.1cm}
\item Tree Structured Group Lasso


\vspace{+.1cm}
\item Overlapping Group Lasso


\vspace{+.1cm}
\item Non-negative Max-heap

\vspace{+.1cm}
\item Sparse Inverse Covariance Estimation

\vspace{+.1cm}
\item Trace Norm Regularized Optimization

\end{list2}


\vspace{-.2cm} The SLEP package is available at
\url{http://yelab.net/software/SLEP/}.


%Currently, the SLEP package is implemented in Matlab, with the
%associated Euclidean projections in C.




\section{\sc Summary}
\begin{list2}

\item Skilled experience in creative research.

\vspace{+.1cm} \item Solid mathematical and statistical knowledge
foundation.

\vspace{+.1cm}\item Strong programming ability. Master C, C++,
Matlab, and SAS.

\vspace{+.1cm}\item Competent communication skill.

\vspace{+.1cm} \item Skilled scientific writing.

\end{list2}


%\section{\sc Referees}
%
%\begin{list2}
%
%
%\item Jieping Ye
%
%Associate Professor
%
%Computer Science and Engineering
%
%School of Computing, Informatics and Decision System Engineering
%
%Arizona State University
%
%Tempe, AZ 85287-8809, U.S.A.
%
%
%Email: jieping.ye@asu.edu
%
%Phone: 1-480-727-7451
%
%Webpage: \url{http://www.public.asu.edu/~jye02/}
%
%\vspace{+.3cm}
%
%\item Songcan Chen
%
%Professor, Director, Pattern Recognition and Neural Computing
%Laboratory
%
%Department of Computer Science \& Engineering
%
%Nanjing University of Aeronautics and Astronautics
%
%Nanjing, Jiangsu 210016, P. R. China
%
%
%Email: s.chen@nuaa.edu.cn
%
%Phone: 86-25-84896477 ext. 12221
%
%
%Webpage: \url{http://parnec.nuaa.edu.cn/}
%
%\vspace{+.3cm}
%
%\item Sudhir Kumar
%
%Professor, Director, Center for Evolutionary Medicine and
%Informatics
%
%The Biodesign Institute
%
%Arizona State University
%
%Tempe, AZ 85287-5301, U.S.A.
%
%
%Email: s.kumar@asu.edu
%
%Phone: 1-480-727-6949
%
%
%Webpage: \url{http://www.kumarlab.net/}
%
%\end{list2}

\end{resume}


\end{document}
